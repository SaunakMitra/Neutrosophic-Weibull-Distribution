\documentclass[12pt,a4paper,oneside]{article}
\usepackage[utf8]{inputenc}
\usepackage{amsmath}
\usepackage{amsfonts}
\usepackage{amssymb}
\usepackage{bigints}
\usepackage{subfigure}
\usepackage{subfig}
\usepackage{float}
\usepackage{multicol}
\usepackage{makeidx}
\usepackage{subcaption}
\usepackage{caption}
\usepackage{lipsum}
\usepackage{graphicx}
\usepackage[dvipsnames]{xcolor}
\definecolor{blue-violet}{rgb}{0.54, 0.17, 0.89}
\usepackage{xcolor}
\usepackage{setspace}
\usepackage{draftwatermark}
\SetWatermarkText{Neutrosophic Extension of Weibull Distribution in Reliability Model}
\SetWatermarkScale{1}
\usepackage[left=2cm,right=2cm,top=2cm,bottom=2cm]{geometry}
\author{Saunak Mitra}
\begin{document}
\textrm{\textit{\textbf{\underline{What is meant by Neutrsophic ?$-$}}}}\\
Neutrosophy (1995) is a branch of philosophy,introduced by Florentin Smarandache in 1998,which studies the origin,nature and scope of neutralities .  Neutrosophy  is an extension of the international movement called paradoxism based on contradictions  in Science , literature and arts (1980). New  Generalization   about several distributions  can be suggested in hopes of expanding  their applicability in field of reliability modeling.This  generalization is based on the  notion of  neutrosophy presented by Smardanche. The analysis  of false or true  statements  having indeterminate ,  neutral   and inconsistent nature is oriented by neutrosophic  logic . Every area has its  neutrosophic component , the indeterminate part , on the mathematical side . Smarandache made the first effort to use the neutrosophic  approach in Statistics ,precalculus  and calculus to cope with imprecision in study variables. As a result , neutrosophic  Statistics have given rise to research topics  that deal with the effect of indeterminacy in Statistical modeling. Neutrosophic  decision- making applications in quality control seem to be very efficient . Neutrosophic principles constitute the basis for neutrosophic logic ,neutrosophic probability ,neutrosophic set  and neutrosophic statistics.\newline\newline
\textrm{\textit{\textbf{\underline{Neutrosophic  Sets :-$-$}}}}\\
Neutrosophic  set is a generalization based on  neutrosophic principles of the intutionistic  set , classical set ,fuzzy set ,paraconsistent set ,dialetheist set ,paradoxist set , and tautological set. \\
Neutrosophic set is a generalization of crisp set ,fuzzy set ,intutuionistic fuzzy set , inconsistent intutionistic  fuzzy set(picture fuzzy set ,ternary fuzzy set),Pythagorean fuzzy set,Fermatean set,q-rung othopair fuzzy set ,spherical fuzzy set  and n-hyperspherical fuzy set.Neutrosophic set has been further extended to  refined neutrosophic set.\newline\newline
\textrm{\textit{\textbf{\underline{Neutrosophic probability :- $-$}}}}\\
It is the generalization of the classical probability and imprecise probability,based on neutrosophy,in which the chance that an event A occurs is t$\%$ true where t varies in the subset T,i$\%$ indeterminate,where i varies in the subset I and f$\%$false where f varies in the subset F. In classical probability,the sum of all space probabilities is equal to 1,while in neutrosophic probability,it is upto 3.\\The function that models the neutrosophic probability of a random variable x is called neutrosophic distribution:NP(x)= (T(x), I(X), F(x) ) , where  T(x) represents the probability that the value x occurs , F(x) represents the probability that the value x does not occurs,and I(X) represents the indeterminate $/$ unknown probability of value x to occur or not.\newline\newline
\textrm{\textit{\textbf{\underline{Neutrosophic statistics :- :- $-$}}}}\\
Neutrosophic statistics is the analysis of events described  by neutrosophic probability .Neutrosophic statistics refers to a set of data,such that the data or a part of it are are indeterminate to some degree,and to methods used to analyze the data. \\
Neutrosophic statistics is  also  a generalization of  interval statistics , because among others , while interval statistics is based on interval analysis ,neutrosophic statistics is based on set analysis(meaning all kinds of sets, not only intervals ).\\
Neutrosophic statistics is more elastic than classical statistics . If all data and inference  methods are determinate , then neutrosophic  statistics coincides with classical statistics.
\end{document}